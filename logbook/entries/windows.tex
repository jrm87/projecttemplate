Most of the code here would use the `*`nix commands, 
but there are ways of implementing in on Windows machines. 
According to Dingle, the two parts of our workflow that are most difficult to implement on Windows
are executing Makefiles and creating symbolic links.

I still need to check for WSL, but I don't think this is correct for symbolic links. 

See here for a \href{https://www.howtogeek.com/16226/complete-guide-to-symbolic-links-symlinks-on-windows-or-linux/}{Guide}:
\begin{enumerate}
    \item Run Command Prompt as admin
    \item Run:
    \begin{enumerate}
        \item **soft** link to file: `mklink Link Target`
        \item **soft** link to directory: `mklink /D Link Target`
        \item **hard** link to file: `mklink /H Link Target`
        \item **hard** link to directory: `mklink /J Link Target`
    \end{enumerate}
\end{enumerate}

\textcolor{red}{WARNING:} If using paths with space, need to use double quotes. 
Example:
\begin{lstlisting}[language=make]
        mklink /J "C:\Link To Folder" "C:\Users\Name\Original Folder"
\end{lstlisting}

\paragraph{Usage:}
I have used them in:
\begin{itemize}
    \item Integrating Github MOC with Dropbox, so Git stores code and Dropbox data and output. 
    \begin{itemize}
        \item Each time the git repository is cloned, for the code to see the data you create a symlink to the corresponding Dropbox data and output folder.
        \item Example:
        \item Data: 
        \begin{lstlisting}[language=make]
            mklink /J "C:\Users\hql910\git\pol_lan\data" "D:\Dropbox\police_language_rep\pol_lan\data"
        \end{lstlisting}
        \item Output: 
        \begin{lstlisting}[language=make]
            mklink /J "C:\Users\hql910\git\pol_lan\output" "D:\Dropbox\police_language_rep\pol_lan\output"
        \end{lstlisting}
    \end{itemize}
\end{itemize}


For WSL, the recommendation is simply installing Ubuntu on Windows Subsystem for Linux (WSL).
Here are instructions:
\href{https://ubuntu.com/tutorials/install-ubuntu-on-wsl2-on-windows-10}{Install Ubuntu on WSL2 on Windows 10}.
You will need to install the Linux versions of all applications, such as R, LaTeX, and Stata.
This is genuine Ubuntu; it will not invoke your Windows executables.
